%%%%%%%%%%%%%%%%%%%%%%%%%%%%%%%%%%%%%%%%%%%%%%%%%%%%%%%%%%%%%
\begin{tframe}{Conclusioni}
\begin{itemize}
\item In conclusione questo lavoro propone dei nuovi approcci al fine di calibrare una curva overnight allo stato dell'arte; per fare ciò propone di:
\end{itemize}
\end{tframe}
%%%%%%%%%%%%%%%%%%%%%%%%%%%%%%%%%%%%%%%%%%%%%%%%%%%%%%%%%%%%%
\begin{tframe}{Conclusioni}
\begin{itemize}
\item In conclusione questo lavoro propone dei nuovi approcci al fine di calibrare una curva overnight allo stato dell'arte; per fare ciò propone di:
     \begin{enumerate}
         \item Eliminare la sovrapposizione di strumenti in curva agganciando l'ultimo strumento spot con il primo strumento forward attraverso una Meta-Quota implicita nel mercato chiamata "Forward Stub"
     \end{enumerate}
\end{itemize}
\end{tframe}
%%%%%%%%%%%%%%%%%%%%%%%%%%%%%%%%%%%%%%%%%%%%%%%%%%%%%%%%%%%%%
\begin{tframe}{Conclusioni}
\begin{itemize}
\item In conclusione questo lavoro propone dei nuovi approcci al fine di calibrare una curva overnight allo stato dell'arte; per fare ciò propone di:
     \begin{enumerate}
         \item Eliminare la sovrapposizione di strumenti in curva agganciando l'ultimo strumento spot con il primo strumento forward attraverso una Meta-Quota implicita nel mercato chiamata "Forward Stub"
         \item Interpolare la curva attraverso un'interpolazione mista che unisca una uno schema lineare per la sezione a breve della curva e uno schema cubico monotono per la sezione a medio/lungo termine.
     \end{enumerate}
\end{itemize}
\end{tframe}
%%%%%%%%%%%%%%%%%%%%%%%%%%%%%%%%%%%%%%%%%%%%%%%%%%%%%%%%%%%%%
\begin{tframe}{Conclusioni}
\begin{itemize}
\item In conclusione questo lavoro propone dei nuovi approcci al fine di calibrare una curva overnight allo stato dell'arte; per fare ciò propone di:
     \begin{enumerate}
         \item Eliminare la sovrapposizione di strumenti in curva agganciando l'ultimo strumento spot con il primo strumento forward attraverso una Meta-Quota implicita nel mercato chiamata "Forward Stub"
         \item Interpolare la curva attraverso un'interpolazione mista che unisca una uno schema lineare per la sezione a breve della curva e uno schema cubico monotono per la sezione a medio/lungo termine.
         \item Stimare la size dei jump al fine di eliminarli prima della calibrazione e aggiungerli nuovamente una volta ultimata.
     \end{enumerate}
\end{itemize}
\end{tframe}
%%%%%%%%%%%%%%%%%%%%%%%%%%%%%%%%%%%%%%%%%%%%%%%%%%%%%%%%%%%%%
\begin{tframe}{Bibliografia}

\begin{thebibliography}{9}

\bibitem{amebianc1} F.M.~Ametrano, M. ~Bianchetti, {\em Everything you always wanted to know about multiple interest rate curve bootstrapping but were afraid to ask}, SSRN, February 2013.\\

\bibitem{amebianc2} F.M.~Ametrano, M. ~Bianchetti, {\em Bootstrapping the illiquidity, multiple yield curves construction for market coherent forward rates estimation},
SSRN, March 2009.\\

\bibitem{amemaz} F.M.~Ametrano, P. ~Mazzocchi, {\em EONIA Jumps and Proper Euribor Forwarding: The Case of Synthetic Deposits in Legacy Discount-Based Systems}, https://speakerdeck.com/nando1970/eonia-jumps-and-proper-euribor-forwarding, December 2014.\\

\end{thebibliography}

\end{tframe}
%%%%%%%%%%%%%%%%%%%%%%%%%%%%%%%%%%%%%%%%%%%%%%%%%%%%%%%%%%%%%
\begin{tframe}

\begin{thebibliography}{9}

\bibitem{burgh} G.~Burghardt, S. Kirshner {\em One good turn}, CME Interest Rate Products Advanced Topics. Chicago: Chicago Mercatile Exchange,2002\\

\bibitem{opengamma} Y.~Iwashita {\em Piecewise Polynomial Interpolations}, Quantitative research, OpenGamma, May 2013.\\

\end{thebibliography}

\end{tframe}
%%%%%%%%%%%%%%%%%%%%%%%%%%%%%%%%%%%%%%%%%%%%%%%%%%%%%%%%%%%%%